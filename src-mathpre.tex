\nocite{8b5861fc}
\nocite{c7f15065}
In this section we recount the mathematics needed to receive an impression of topological insulators. The mathematical branches relevant for the theory of topological insulators
\begin{enumerate}
\item[$\bullet$]
  (functional) analysis
\item[$\bullet$]
  homotopy theory
\end{enumerate}
The former is relevant to describe the quantum mechanical aspect and the latter the topological. And while we assume the reader to be familiar with functional analysis (i.p. spectral theory and the mathematics of quantum mechanics) and basic homotopy theory (see e.g. \cite{catctxphy} or the standard reference \cite{8b5861fc}) we do not require the reader to be familiar with more advanced homotopy theory. Most notably, we do not assume any knowledge on topological K-theory. Topological K-theory is the main tool for a rough understanding of topological phases and it is what this section is about. Our presentation is essentially an abstraction of \cite{c7f15065}. Accordingly, we have subsections on categories, the Grothendieck group and vector bundles needed to eventually treat topological K-theory in the last subsection. Note that our treatment of these topics is very high-level and algebraic mostly glossing over geometrical aspects. This is suffices our needs but eventually you may wish to go with something like \cite{c7f15065} for a deeper understanding. In fact, characteristic classes - i.p. Chern classes - as presented there are important for a thorough understanding of our account of topological insualtors in section \ref{sec:topins}. This also involves some differential geometry but for a high-level picture on the topic both topics can be neglected.
\\
\begin{rem}
\label{rem:style}
Before getting started a few words about the mathematical style: 
\begin{enumerate}
\item[$\bullet$]
  Mathematical definitions use \textbf{boldfaced type} to indicate the term being defined. In later sections we will sometimes encounter new terms that we cannot define here for different reasons. In this case we will use \textit{italic type} instead of boldfaced type and the reader is referred to other sources. In these preliminaries we usually give an example to each definition for the sake of comprehension. However, most often we do not give a proof of the propositions we make. If at all, we only sketch one. 
\item[$\bullet$]
  $\mathsf{A}$ stands for both a finite and infinite index set, that is, an arbitrary set with specified cardinality. $\mathbb{K}$ stands for either the real numbers $\mathbb{R}$ or the complex numbers $\mathbb{C}$ and $(\cdot \vert \cdot)$ denotes the scalar product. Following the convention we use the term \textit{space} for a set together with some \textit{structures} (e.g. topology, vector space structure, \ldots) and \textit{subspace} means a subset of the space's set together with the structure it can \textit{inherit}. In the context of spaces we redefine \textbf{map} to mean spaces-preserving function for the sake of brevity. That is, map in the context of topological spaces means continuoous function, for example.
\end{enumerate}
\phantom{proven}
\hfill
$\square$
\end{rem}
