\nocite{eda79af1}
\nocite{fd8cbb36}
\nocite{1706ae98}
The 2016 Nobel laureates in physics are Thouless, Haldane and Kosterlitz {\glqq}for the theoretical discoveries of topological phase transitions and topological phases of matter{\grqq} as officially stated by the Nobel Foundation. Shortly after von Klitzing in 1980 observed a quantized Hall conductance of a two dimensional system of electrons at nearly zero temperature in a perpendicular magnetic field, the 2016 Nobel laureates made it to show that these very robust quantization does not merely have its origin in ordinary quantum mechanics but rather in the topology of the spectrum.
\\
Von Klitzing found for the Hall conductance that
\begin{align*}
  \sigma
  &=
  \frac{e^{2}}{h}
  \nu
  ,\qquad
  \nu
  \in
  \mathbb{N}
\end{align*}
where $e$ is the elementary charge and $h$ is Planck's constant. $\nu$ is an integer depending monotonously on the strength of the magnetic field and is zero for a disappearing magnetic field. Consequently, one calls the effect above the \textit{integer quantum Hall effect (IQHE)}. It turned out that $\nu$ was well-known topological invariant of the spectrum related to K-theory making the IQHE the historically\footnote{compare \cite{1706ae98}} first type of a topological phase distinction. If the IQHE is exhibited in a solid insulator the insulator is called topological insulator.
\\
The theory of topological insulators is a young field of research that utilizes many mathematical theories which is probably the reason why there are not so many convenient ways to get in touch with it - for the common undergraduate student, at least. The paper's main purpose is to make up for this serving as an accessible introduction to topological insulators.
\\
The paper is structured as follows
\begin{enumerate}
\item[$\bullet$]
Section \ref{sec:mathpre} discusses the mathematical background needed to get a rough understanding of toplogical insulators. The ultimate ambition of section \ref{sec:mathpre} is to tool up the reader with topological K-theory which, in some limited cases, allows describing topological insulators.
\item[$\bullet$]
Section \ref{sec:topins} tries to convey a high-level picture of topological insulators. We will start out saying what we mean by insulator precisely. After that, we briefly outline the general K-theoretic classification of topological phases. The section culminates in the somewhat famous periodic table for topological insulators proposed in \cite{eda79af1} by Kitaev as well as the high-level presentation of the IQHE and more general Chern insulators which are contained in this periodic table.
\item[$\bullet$]
Section \ref{sec:kmm} - the last section - concerns the Kane-Mele model on a lower-level. It is a model for another kind of (two-dimensional) topological insulators contained in Kitaev's periodic table. To put it in a nutshell we use methods similar to that of section \ref{sec:topins} to describe the model and to find a topological invariant like the integer in the IQHE. It will be explicit enough to implement it as opposed to the dicussion in section \ref{sec:topins}.
\end{enumerate}
